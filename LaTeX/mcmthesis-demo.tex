\documentclass[12pt]{mcmthesis}
\mcmsetup{CTeX = false,   % 使用 CTex 套装时,设置为 true
        tcn = 2616546, problem = B,% 队伍控制号码,接受一个字符串作为值;选题,接受一个字符串作为值;
        sheet = true, %为真时将输出摘要页,否则不输出;默认为 true。
        color = red,  %设置控制页的题目号的颜色
        titleinsheet = true, %为真时将在摘要页输出标题,否则不输出;默认为 false。
        keywordsinsheet = true,%为真时将在摘要页输出关键字,否则不输出;默认为 false。
        titlepage=false,%为真时将输出标题页,否则不输出;默认为 true。
        abstract=true}%为真时将在标题页输出摘要和关键词,否则不输出;默认值为 true。
\usepackage[T1]{fontenc}
\usepackage{newtxtext} % 正文使用 Times 风格
\usepackage{newtxmath} % 数学公式也匹配 Times 风格
%\usepackage{palatino}  %控制正文字体,若是不喜欢可以注释掉。
\usepackage{lipsum}
\usepackage{amsmath}
\usepackage{amssymb}
\usepackage{subfig}
\usepackage{float}%避免浮动体宏包
\usepackage{indentfirst}
\newcommand{\itemEq}[1]{%
	\begingroup%
	\setlength{\abovedisplayskip}{0pt}%
	\setlength{\belowdisplayskip}{0pt}%
	\parbox[c]{\linewidth}{\begin{flalign}#1&&\end{flalign}}%
	\endgroup}%itemiz里面套equation
\setlength{\parindent}{2em}
\title{Establishing a lunar colony using hybrid transportation methods}
%采用混合运输方式建立月球殖民地
\author{A \and B \and C}
\date{\today}

\makeatletter
\renewcommand*\l@section{\@dottedtocline{1}{12pt}{12pt}}
\newcommand{\upcite}[1]{\textsuperscript{\textsuperscript{\cite{#1}}}}
\makeatother

\begin{document}
\begin{abstract}
	\small

This paper addresses the problem of establishing lunar colonies using a space elevator system, which essentially constitutes a cost minimisation linear programming problem under cargo capacity constraints. A mathematical model for mixed-material transport has been developed, with optimal time as the objective. The specific research content is as follows:
%本文探讨了利用太空电梯系统建立月球殖民地的问题,该问题本质上是在货物承载能力约束下的成本最小化线性规划问题。已建立混合物料运输的数学模型,以最优运输时间为目标函数。具体研究内容如下

First, we coupled a continuous flow system (space elevator) with a discrete batch transport system (rocket launch). Subject to the constraint of total cargo volume, we developed a hybrid transport model capable of addressing multiple scenarios, resolving at least the following four problems:
%首先,我们将连续流运输系统(太空电梯)与离散批量运输系统(火箭发射)相结合。在总货物体积约束条件下,我们开发出一种混合运输模型,能够应对多种场景,至少解决了以下四个问题

\textbf {For Question 1:} Under ideal conditions, \textbf{the hybrid mode (space elevator and rocket) has obvious advantages} — it reduces transport time to about $70$ years and keeps costs moderate, outperforming either a pure elevator or a pure rocket solution. For the time and cost of other transportation methods, you can refer to Table $1$.

\textbf{For Question 2:} Took the initial scenario a step further by factoring in real-world system availability, represented by a disturbance coefficient. Once this practical constraint was added, the mission timelines extended to $196$ years, $113.38$ years, and $71.83$ years, with corresponding costs of $1.015\times10^{11}$, $5\times10^{11}$, and $3.59\times10^{11}$,you can refer to Figure 2 for details.The hybrid method still came out on top for speed, with its cost falling between the other two options. An important shift here is that the pure space elevator’s advantage shrank noticeably once disturbances were part of the equation. Under these more realistic, non-ideal conditions, the hybrid approach clearly stands out as the best balance of time and cost.

\textbf{For Question 3:} After incorporating life support and environmental objectives, \textbf{the hybrid model was once again proven to be the fastest}, requiring only $0.56$ years. The transportation costs and times for other methods and conditions can be found in Tables $3$ and $4$. At the same time, a key finding is that in this hybrid framework, assigning more cargo to the elevator can \textbf{significantly reduce the total mission cost}.

\textbf{For question 4:} In most cases, the hybrid approach really does strike the best balance between cost and time. But when we look at environmental impact, the differences between the three options are pretty dramatic.In terms of total CO₂e emissions after building the lunar colony and running a year of water resupply:Using only the Galactic Port (“space elevator”) comes out to around 1,007.85 million tonnes.Traditional rockets alone emit a massive 322,511.20 million tonnes.The fastest hybrid option lands at 202,027.98 million tonnes.So even though the hybrid method is already the fastest, it also beats rockets hands-down on emissions, while still being way quicker than relying solely on the elevator. \textbf{That tells us the space elevator system isn’t just fast — it’s also a much cleaner alternative}. For the detailed breakdown, you’d want to check Table 5 (ideal case) and Table $6$ (with disturbance factors).

The model system constructed herein features \textbf{efficient solution methods and clear hierarchical structure}. The derived strategies yield effective outcomes in evaluating space transport scheme performance and optimising parameters, providing \textbf{not only efficient solutions for transport challenges but also demonstrating practical feasibility for real-world applications}.
%本文构建的模型系统兼具高效求解方法与清晰的分层结构。所推导的策略在评估空间运输方案性能及优化参数方面成效显著,不仅为运输难题提供了高效解决方案,更展现出在实际应用中的可行价值.  

\begin{keywords}
	\footnotesize
Hybrid transport model; Space elevator; Resource allocation; Linear programming
%混合运输模型;太空电梯;资源分配;线性规划
\end{keywords}
\end{abstract}
 \maketitle

\tableofcontents
\clearpage
\normalsize
\section{Introduction}%问题背景
\subsection{Assumptions and Symbols}
The space elevator system is a concept for the construction of a lunar colony with a projected capacity of $100,000$ people after the year $2050$. This undertaking would necessitate the procurement and transportation of approximately one billion tonnes of construction materials. Two primary methods have been proposed for the conveyance of these materials. Firstly, the utilisation of a future space elevator system, capable of transporting $179,000$ tonnes annually without compromising the integrity of the atmosphere. Secondly, the employment of contemporary heavy-lift rockets, with a capacity ranging from $100$ to $150$ tonnes, launched from ten distinct locations.
%结合太空电梯系统计划在2050年后要建一个能住10万人月球殖民基地,光建筑材料就要大约10亿吨。运输材料有两个方法:一个是使用每年能运17.9万吨,而且不污染空气未来的太空电梯系统,另一个是用每次能运100到150吨现在的重型火箭从十个发射场发射。

%\begin{figure}[H]
	%\small
	%\centering
	%\includegraphics[width=5cm,height=10.6cm]{picture1.png}
	%\caption{Transportation Process}
 %运输流程
	%\label{tab:2}
%\end{figure}

Our task is to build a mathematical model to calculate the cost and time required to transport these materials to the Moon starting from 2050. We need to consider three scenarios: using only a space elevator; using only traditional rockets; and using both together. The model should take into account both perfect conditions and conditions with disturbances. After the base is built, we also need to calculate the additional cost and time for operating water transport. Additionally, we need to assess the impact of the transportation methods on the Earth's environment and figure out ways for the model to minimize environmental damage. Finally, we need to write a decision recommendation letter to the MCM.
%当前任务需要比较这两种运输方式所需的成本和时间。目标是建立数学模型,用于计算自2050年起将这些物资运送至月球的成本与时长。需考虑三种情景:仅使用太空电梯;仅使用传统火箭;以及两种方式并行使用。该模型不仅需涵盖理想条件,还需评估潜在问题对成本和时间线的影响。建立基础模型后,需测算这10万居民的年用水需求,并评估水资源运输的额外成本与时间消耗。此外,必须评估运输方式对地球环境的影响,并设计模型以最大限度减少环境损害。最后,需向月球管理委员会提交决策建议书。
\subsection{Ourwork}

\begin{figure}[H]
	\small
	\centering
	\includegraphics[width=12cm,height=9.6cm]{picture4.png}
	\caption{Flowchart}
\end{figure}
In accordance with the specified scenario, the procedural sequence has been delineated in Figure 1. This figure meticulously delineates the precise procedures that have been followed by both modes of transport in the Earth-Moon material delivery operation. This established sequence of thought processes is characterised by its rigorous nature, with all subsequent stages strictly adhering to the stipulated flowchart.
%根据情景我们绘制了图1流程,其中详细说明了两种运输方式在地月物资运送方面所经过的具体流程,建立了严谨的思维流程网络,后续的部分都是严格按照流程图进行操作的。

\section{Assumptions and Symbols}
\subsection{Model Hypothesis}
%模型假设
It is assumed that the average launch cost from a rocket launch site to the Moon is $\$5,000$ per kilogramme\upcite{1}. By comparison, the transport cost under the space elevator project is $ \$500$ per kilogramme\upcite{2} (delivered to Galaxy Port). According to the Tsiolkovsky rocket equation, the cost of transporting cargo from Galaxy Port to the Moon is calculated to be $\$515$ per kilogramme.
%假设从火箭发射场运送货物至月球的平均成本为每公斤5000美元。相比之下,太空电梯项目下的运输成本为每公斤500美元(运抵银河港)。根据齐奥尔科夫斯基火箭方程计算,从银河港向月球运输货物的成本为每公斤515美元。

It is hypothesised that the Earth Current rocket launch sites have the capacity to operate at full capacity\upcite{3}\upcite{4}, with rockets being launched twice daily\upcite{5} and Falcon Heavy rockets being selected for material transport.
%据推测,地球电流火箭发射场具备满负荷运转的能力,每日可执行两次火箭发射任务,并选定猎鹰重型火箭承担物资运输任务。

In the ideal scenario, both alpha and beta are equal to $1$ (a fact that has not been explicitly demonstrated in the formula). However, under more optimistic conditions, $\alpha$ is set at $0.95$, while $\beta$ is set at $0.98$. 
%在理想情况下,阿尔法系数与贝塔系数均等于1(该事实未在公式中明确体现)。然而在更乐观的条件下,阿尔法系数设定为0.95,贝塔系数设定为0.98;



\subsection{Symbols and Definitions and Unit}
%符号与定义及单位
\begin{table}[H]
\centering
\renewcommand{\arraystretch}{1.6} 
%\caption{Symbol Legend}%符号说明
	\begin{tabular}[cp{0.9\textwidth}]{cc}
		\toprule[2pt]
		Symbols & Description\\
		\toprule[2pt]
		%\midrule
		$M_{total}$ & Total mass of materials required for the corresponding scenario   \\
        $M_i$ & Quality allocated to the system in the corresponding context $i$ ($i \in \{a, b\}$)  \\
        $R_i$ & Annual transport rate of system $i$ ($i \in \{a, b\}$)  \\
        $T_i$ & Total duration for scenario $i$ ($i \in \{a, b, c\}$)  \\
        $D_i$ & Total cost for scenario $i$ ($i \in \{a, b, c\}$)  \\
        $f_i$ & Operational frequency of system $i$ ($i \in \{a, b\}$)  \\
        $N_{rk}$ & Total number of rocket launches required  \\
        %$\beta$ & Success probability of a single rocket launch ($0 < \beta \le 1$) & Dimensionless \\
        $P_i$ & Actual annual capacity of system $i$ under imperfect conditions ($i \in \{a, b\}$)  \\
        $c_i$ & Unit transport cost of system $i$ under imperfect conditions ($i \in \{a, b\}$)  \\
        $T_i$ & Transport duration for scenario $i$ under imperfect conditions ($x \in \{a, b, c\}$) \\
        $C_i$ & Total transport cost for scenario $i$ under imperfect conditions  \\
        $x$ & The load capacity is within the range of 100 to 150 tonnes. \\
		\bottomrule[2pt]
	\end{tabular}
	\label{tab1} 

\end{table}

\section{Problem Analysis and Model Development and Solution}
%问题的分析和模型的建立与求解
Following a comprehensive review of the pertinent documentation, it has been determined that the financial and temporal implications of transporting goods vary according to the selected mode of conveyance. Consequently, in instances where the quantity of goods is fixed and the mode of transport is predetermined, the total cost should also be fixed. Utilising this information, we can establish the parameters for examining three potential scenarios for each of the two transport methods.
%经对相关文件进行全面审查,已确定货物运输的财务与时间影响因所选运输方式而异。因此,当货物数量固定且运输方式预先确定时,总成本也应保持固定。基于此信息,我们可以为两种运输方式各自的三种潜在情景建立分析参数。
\begin{itemize}
\item \textbf{{Solution for scenario i}}
\end{itemize}

It is estimated that each Galaxy Port system possesses the capacity to facilitate the transportation of $179,000$ tonnes of cargo from Earth to the Moon on an annual basis. This two-stage transportation process is deemed to be complete only upon the successful delivery of the shipment from the Galaxy Port to the Moon. Consequently, the annual frequency of rocket launches within the confines of the Galaxy Port system can be determined as follows
%据估算,每个银河港系统具备每年将179,000吨货物从地球运送至月球的能力。该两阶段运输流程仅在货物从银河港成功运抵月球后方视为完成。因此,银河港系统内部的年度火箭发射频率可按以下方式确定

Under ideal conditions:
%在完美条件下
$$
    f_a=\frac{17900}{x}
$$

In light of the disturbance factors present in real-world scenarios, the disturbance coefficient alpha is hereby introduced. Consequently, the annual emission frequency is determined as follows
%鉴于现实场景中存在的干扰因素,特此引入干扰系数α。据此,年度排放频率按下列方式确定

$$
    f_{a1}=\alpha f_a\frac{17900\alpha}{x}
$$

The time required for its transportation is (when the parameter known as $\alpha$ is set to one, i.e. under conditions deemed ideal)
%其运输所需时间为(当参数α设为1时,即在理想条件下)
\begin{equation}
    T_{a}=\frac{M_{total}}{3f_ax}=\frac{M_{total}}{537000\alpha}          
\end{equation}

The financial requirements for this mode of transport have been addressed
%该运输方式的资金需求已得到解决
\begin{equation}
    D_a=M_{total}\times(M+m)                    
\end{equation}

\begin{itemize}
\item \textbf{{Solution for scenario ii}}
\end{itemize}

It is estimated that a rocket launch site is capable of conducting two launches per day. Therefore, the annual launch frequency for a rocket launch site is as follows
%据估计,火箭发射场每天可进行两次发射。因此,火箭发射场的年发射频率如下
$$
    f_b= 720
$$

In order to account for disturbances present in real-world scenarios, the disturbance coefficient $\beta$ is hereby introduced. The annual emission frequency is thus given by
%为考虑现实场景中的扰动因素,特此引入扰动系数β。由此,年度排放频率可表示为
$$
    f_{b1}=\beta f_b
$$

The time required for this mode of transport is as follows
%此种交通方式所需时间如下
\begin{equation}
    T_b=\frac{M_{total}}{10x\beta f_b}                  
\end{equation}

The corresponding costs required shall satisfy the following conditions
%所需的相应成本应满足以下条件
\begin{equation}
    D_b=M_{total}\times5000     
\end{equation}

\begin{itemize}
\item \textbf{{Solution for scenario iii}}
\end{itemize}

Assuming that the Solution for scenario i is equally successful as that for scenario b, the following equation is obtained
%假设场景a的解决方案和场景b的解决方案同样成功,则得到以下方程
$$
    M_a+M_b=M_{total}
$$

n the context of integrated transportation systems involving a combination of two or more modes, there exists%在涉及两种或多种交通方式组合的综合交通系统背景下,存在着
\begin{equation}
T_c=max\{\frac{M_{total}}{3f_{a1}x}\frac{M_a}{M_{total}},\,\frac{M_{total}}{10f_{b1}x}\frac{M_b}{M_{total}}\}
=max\{\frac{M_a}{3\alpha f_ax},\,\frac{M_b}{10\beta f_bx}\}                    
\end{equation}

The cost of this mode of transport shall satisfy the following criteria%此种运输方式的成本应满足以下标准
\begin{equation}
     D_c=D_a+D_b=M_a\times(M+m)+M_b\times5000 
\end{equation}

\begin{figure}[H]
	\small
	\centering
	\includegraphics[width=11.6cm,height=6.5cm]{picture3.png}
	\caption{10 year cost schedule for lunar colonisation}
 %月球殖民地10年成本时间表
\end{figure}
It is hypothesised that the rocket will have a payload capacity of $125$ tonnes and will be operational for a period of $10$ years following completion. The costs and time requirements will adhere to the relationship depicted in the diagram. When both transport methods are employed concurrently, the cost will fall between the two and the time consumption will be negligible.
%据推测,该火箭将具备125吨的有效载荷能力,并在完成建造后可运行10年。其成本与时间需求将遵循图示关系。当两种运输方式同时采用时,成本将介于两者之间,而时间消耗则可忽略不计。
\section{Use for several specific issues}
\subsection{Performance Evaluation of Transport Schemes and Multi-Criteria Decision-Making}%运输方案的效能评估与多准则决策

\subsubsection{Problem Analysis}

In this context, it can be deduced that both $\alpha$ and $\beta$ are equal to one, and neither is reflected in the following expression.
%在此背景下,可推断出$\alpha$和$\beta$均等于1,且二者均未在下列表达式中体现。
\begin{itemize}
\item \textbf{{Solution for scenario i}}
\end{itemize}
%问题分析

It is evident that each Galactic Port is capable of handling a sufficient annual cargo volume to transport $179,000$ metric tonnes of material. Consequently, the time required to deliver the necessary materials can be calculated. It should be noted that Galactic Port shipments involve two distinct stages with differing costs. Therefore, the cost for Scenario a should be the sum of both stages' costs.
%显然,每个银河港口每年可处理的货物量足以运输179,000公吨物资。因此,所需物资的运输时间可据此计算。需注意的是,银河港口运输涉及两个成本不同的独立阶段。故方案a的运输成本应为两个阶段成本之和。

The duration of transportation when utilising solely the space elevator system is as follows
%仅使用太空电梯系统时的运输时长如下
$$
T_a = \frac{M_{total}}{R_a} 
$$

The total cost of building a space elevator is the cost of the elevator itself from Earth to the Moon, and the cost of the rocket from the Moon back to Earth. If we use a soft landing material, the cost is calculated like this
%建造太空电梯的总成本包括从地球到月球的电梯本身成本,以及从月球返回地球的火箭成本。若采用软着陆材料,成本计算方式如下
$$
D_a =  1015M_{total} 
$$
\begin{itemize}
\item \textbf{{Solution for scenario ii}}
\end{itemize}

In consideration of the ten rocket launch sites currently in operation on the Earth, the launch frequency of each rocket, and the payload capacity of the latter, the time required for the transportation of materials can be calculated. With the launch frequency and required time as known constants, the number of rocket launches can be determined. This, in turn, enables the calculation of the cost of transporting materials using only traditional rockets.
%考虑到地球上目前运行的十个火箭发射场、每枚火箭的发射频率及其有效载荷能力,可计算出物资运输所需的时间。当发射频率和所需时间作为已知常量时,即可确定火箭发射次数。进而仅通过传统火箭运输物资的成本便可计算得出。

It is imperative to note that when utilising the Earth Current rocket launch site in isolation, the designated transportation time is as follows
%必须注意的是,当单独使用地球电流火箭发射场时,指定运输时间如下
$$
T_b = \frac{M_{total}}{R_b} 
$$

The total number of rocket launches is
%火箭发射总数为
$$
N_{rk}=10\times f_b\times T_b
$$

The transportation cost using solely the Earth Current launch site can be calculated as follows
%仅使用地球电流发射场时的运输成本可按以下方式计算。
$$
D_b = M_{total} \times 5000
$$
\begin{itemize}
\item \textbf{{Solution for scenario iii}}
\end{itemize}

The hybrid approach combines two transportation methods, with both schemes operating concurrently to convey specific quantities of goods. As the allocation of goods differs between the two schemes, the duration of the hybrid operation is determined by the scheme that requires the longest duration. The cost of the hybrid scheme is determined by the specific quantities of goods transported by each scheme.
%混合运输方式结合了两种运输方案,两者同时运作以运输特定数量的货物。由于两种方案的货物分配存在差异,混合运输的持续时间取决于所需时间最长的方案。混合运输方案的成本则由每种方案运输的具体货物数量决定。

In order to initiate simultaneous transport operations, it is necessary to combine both approaches. The 100 million tons of cargo are divided into $N_a$ and $N_b$. From the time required to transport 100 million tons as determined by To scenarios a, it is possible to derive the time needed to transport $N_a$ tons. Similarly, by applying To scenarios b, it is possible to calculate the time required to transport $N_b$ tons of material. These times are respectively
%为启动同步运输作业,必须将两种方案结合实施。1亿吨货物分为$N_a$吨和$N_b$吨两部分。根据方案a确定的运输1亿吨所需时间,可推算出运输$N_a$吨所需时间。同理,通过方案b可计算出运输$N_b$吨物料所需时间。具体时间分别为

$$
W_a=\frac{R_a}{R_a+R_b}M_{total}\,\,\,\,\,\,\,\,\,    W_b=\frac{R_b}{R_a+R_b}M_{total}        
$$

By combining the linear inequalities, the total transport time can be derived as%通过合并线性不等式,可推导出总运输时间为

$$
T_c = \frac{M_{total}}{R_a + R_b}      
$$

As demonstrated in the accompanying diagram, in order to minimise the total transport time, it is necessary for both scenarios to commence simultaneously and conclude their respective transport tasks concurrently.The total cost of the hybrid model is derived by combining the cost calculation formulas for Scenario A and Scenario B as follows
%如附图所示,为最大限度缩短总运输时间,两种方案必须同时启动并同步完成各自的运输任务。混合模型的总成本通过结合方案A与方案B的成本计算公式得出,具体如下
$$
D_c = T_c \cdot (1015R_a + 5000R_b ) = M_{total} \times \frac{1015R_a  + 5000R_b }{R_a + R_b}
$$

\subsubsection{ Solving models}%求解模型
Substituting the scenario into all the above equations 
$M_{total}=10^8$ ,\,$x=125$ ,\,$\alpha=1$ ,\,$\beta=1$ .
%将该情景代入上述所有方程 $M_{total}=10^8$ ,\,$x=125$ ,\,$\alpha=1$ ,\,$\beta=1$ .
\begin{table}[H]
\centering
\renewcommand{\arraystretch}{1.6} 
\caption{Time Cost Statistics }%问题一时间成本统计表
	\begin{tabular}[cp{0.9\textwidth}]{ccc}
		\toprule[2pt]
		proposal &   Delivery time(years) & Delivery cost(\$)\\
		%\midrule
        \toprule[2pt]
		\textbf{{Solution for scenario i}} & $186.22$ & $1.015\times10^{11}$\\
        \textbf{{Solution for scenario ii}} & $111.11$ & $5.000\times10^{11}$ \\
        \textbf{{Solution for scenario iii}} & $69.59$ & $3.510\times10^{11}$ \\        
		\bottomrule[2pt]
	\end{tabular}
	\label{tab1} 
\end{table}
\begin{figure}[H]
	\small
	\centering
	\includegraphics[width=11.6cm,height=6.5cm]{picture7.png}
	\caption{Relationship between space elevator allocation, time, and cost}
\end{figure}
\subsubsection{Analysis of Results}%结论分析
\begin{figure}[H]
	\small
	\centering
	\includegraphics[width=11.6cm,height=6.5cm]{picture2.png}
	\caption{Capacity allocation ratio}
\end{figure}
We can see that the larger the payload, the heavier the mission the rocket has to undertake.Assuming that the unit transport cost of the space elevator is lower than that of conventional rockets, it is demonstrated that the Galaxy Port system alone yields the lowest total cost, followed by the hybrid approach, with the traditional rocket-only solution being the most expensive.

Should establishing a lunar colony as swiftly as possible after $2050$ be the primary objective, it is recommended that the hybrid transport scheme be selected. This offers the optimal timeline, enabling the delivery of all materials within approximately 70 years, while its cost lies between the two alternatives, presenting favourable overall feasibility.

Therefore, under ideal conditions, the hybrid transport model combining the Galaxy Port system with conventional rockets is recommended as the preferred approach for establishing a $100,000$-person lunar colony.
%假设太空电梯的单位运输成本低于传统火箭,分析表明银河港系统能实现最低总成本,混合运输方案次之,而传统纯火箭方案成本最高。
%若首要目标是在2050年后尽快建立月球殖民地,则推荐采用混合运输方案。该方案提供最优时间表,可在约70年内完成全部物资运输,其成本居于两种方案之间,整体可行性较佳。
%因此在理想条件下,建议采用银河港系统与传统火箭相结合的混合运输模式,作为建立10万人月球殖民地的首选方案。
%---------------------------------------------------------------------
\subsection{Construction of hybrid transportation models based on different scenarios}%基于不同场景的混合交通模型构建
\subsubsection{Problem Analysis}%问题分析
The crux of Question 2 pertains to the quantitative evaluation of the impact of system reliability on the efficacy of transport solutions. It is imperative to integrate reliability parameters into the original model to analyse the variation in transport time and costs across three scenarios (Galaxy Port only, Rocket only, and hybrid solution) under varying failure levels. This will establish a foundation for effective project risk management.
%问题2的核心在于定量评估系统可靠性对运输方案效能的影响。必须将可靠性参数整合至原始模型中,以分析三种场景(仅银河港方案、仅火箭方案及混合方案)在不同故障率下的运输时间与成本变化。此举将为有效的项目风险管理奠定基础。
%\subsubsection{Establishing a Multimodal Transport Model for Different Scenarios}
%建立适用于不同场景的多式联运模型

The core idea is to introduce a coefficient ($0 < coeff \le 1$) that represents the "uptime" or "success rate," which effectively reduces capacity and increases unit cost.

 \textbf{Galactic Harbor ($\alpha$): Represents availability (affected by tether sway, maintenance)}.
 
 \textbf{Rocket System ($\beta$): Represents success probability (affected by launch failures)}.

%模型逻辑与定义
%核心思想是引入一个系数(0 < coeff ≤ 1),该系数代表"运行时间"或"成功率",能有效降低运力并增加单位成本。
%• 银河港(α)代表可用性(受系绳摆动、维护影响)。
%• 火箭系统 (β)代表成功概率(受发射失败影响)。
\begin{itemize}
    \item \textbf{Solution for scenario i}:
\end{itemize}
 
 Galactic Harbor (with disturbance $\alpha$)The disturbance reduces the effective rate and increases the unit cost inversely.
 %方程分解与转录系统A银河港(含扰动因子$\alpha$)该扰动因子会降低有效速率,并反向增加单位成本。

Effective Annual Capacity
%有效年产能
$$
    P_{a} = \alpha \cdot R_a
$$

Transport Time
%运输时间
$$
    T_a = \frac{M_{total}}{P_{a}}
$$

Unit Cost (Disturbed)
%单位成本(扰动)
$$
    c_a = \frac{D_a}{\alpha}
$$

Total Cost
%总成本
$$
    c_a = M_{total} \cdot c_a
$$
\begin{itemize}
    \item \textbf{Solution for scenario ii}: 
\end{itemize} 

Rocket System (with disturbance $\beta$)
%系统B火箭系统(含扰动$\beta$)

Effective Annual Capacity
%有效年产能
$$
    P_{b} = \beta \cdot R_b
$$

Unit Cost (Disturbed)
%运输时间
$$
    c_b = \frac{D_b}{\beta}
$$
\begin{itemize}
\item \textbf{Solution for scenario iii}:
\end{itemize}
 
 Hybrid CorrectionThe hybrid model optimizes by equating the transport times of both systems ($T_c$) to ensure simultaneous completion.
%系统C混合校正混合模型通过使两个系统的运输时间($T_c$)相等来优化,以确保同时完成。

Constraint%约束条件
$$
    \frac{M_{a}}{P_{a}} = \frac{M_{b}}{P_{b}} = T_c
$$

Total Time%总时间
$$
    T_c = \frac{M_{total}}{P_{a} + P_{b}}
$$

Load Allocation%负载分配
$$
    M_a = \frac{P_a}{P_a + P_b} \cdot M_{total}\,\,\,\,\,\,\,M_b = \frac{P_b}{P_a + P_b} \cdot M_{total}
$$

The total cost of a hybrid system under disturbance conditions can be expressed as%在干扰条件下,混合系统的总成本可表示为
$$
    C_{total} = \left( \frac{P_{a} \cdot M_{total}} {P_{a} + P_{b}} \right) \cdot \frac{D_a}{\alpha} + \left( \frac{P_{b} \cdot M_{total}}{P_{a} + P_{b}} \right) \cdot \frac{D_b}{\beta} 
$$

\subsubsection{Solving the model}
In order to quantify the impact of imperfect conditions, two sets of reliability parameter scenarios are established for analysis purposes. The first scenario is optimistic, with relatively high system reliability ($\alpha=0.95$), and the second is conservative, with relatively low system reliability ($\beta=0.90$). The specific parameter values and their rationale are as follows:
Optimistic Scenario: As an emerging yet mature technology, Galaxy Port exhibits high availability ($\alpha=0.95$); rocket launch technology, having developed over decades, is highly reliable ($\beta=0.98$).
%为量化不完美条件的影响,建立了两组可靠性参数情景供分析使用。第一种情景为乐观情景,系统可靠性相对较高($\alpha=0.95$);第二种为保守情景,系统可靠性相对较低($\beta=0.90$)。具体参数值及其依据如下
%乐观情景银河港作为新兴成熟技术,具备高可用性(α=0.95);火箭发射技术经数十年发展,可靠性极高(β=0.98)。

Substituting the scenario into all the above equations: 
$M_{total}=10^8$ ,\,$x=125$ ,\,$\alpha=0.95$ ,\,$\beta=0.98$ .
%将该情景代入上述所有方程 $M_{total}=10^8$ ,\,$x=125$ ,\,$\alpha=1$ ,\,$\beta=1$ .
\begin{table}[H]
\centering
\renewcommand{\arraystretch}{1.6} 
\caption{Time cost statistics }
	\begin{tabular}[cp{0.9\textwidth}]{ccc}
		\toprule[2pt]
		proposal &   Delivery time(years) & Delivery cost(\$)\\
        \toprule[2pt]
		%\midrule
		\textbf{{Solution for scenario i}} & $196.02$ & $1.015\times10^{11}$\\
        \textbf{{Solution for scenario ii}} & $113.38$ & $5.000\times10^{11}$ \\
        \textbf{{Solution for scenario iii}} & $71.83$ & $3.539\times10^{11}$ \\        
		\bottomrule[2pt]
	\end{tabular}
	\label{tab1} 
\end{table}
\begin{figure}[H]
	\small
	\centering
	\includegraphics[width=13.4cm,height=6.6cm]{picture5.png}
	\caption{Cost and time comparison}
\end{figure}
\subsubsection{Analysis of Results}
In circumstances that deviate from ideal conditions, the conclusions derived from Problem 1 are subject to modification. When utilising the space elevator as the sole means of transportation, there is a considerable fluctuation in the time cost, resulting in instability. Conversely, the hybrid strategy demonstrates minimal variation, ensuring consistent delivery times. Consequently, our solution remains unaltered.
%在偏离理想条件的情况下,问题1得出的结论需作相应调整。若仅采用太空电梯作为运输手段,时间成本将出现显著波动,导致运输时效不稳定。反之,混合策略展现出极小波动性,可确保交付时间的稳定性。因此,我们的解决方案保持不变。
%------------------------------------------------------------------
\subsection{Analysis of Annual Water Supply for Lunar Colonies}
%月球殖民地年度供水分析
\subsubsection{Problem Analysis}%问题分析
In order to resolve the issue under discussion, it is first necessary to recognise the necessity of ensuring the availability of water prior to the arrival of the colonists, thus avoiding any risk of water scarcity. Secondly, in order to make an informed decision, it is essential to calculate a table detailing the relationship between water loss, delivery costs, and time required after the lunar colony is established, once the annual water supply for $100,000$ settlers has been delivered and they commence living there.
%为解决当前议题,首要任务是认识到必须确保殖民者抵达前水源的可用性,从而规避任何缺水风险。其次,为做出明智决策,必须编制详细表格,阐明月球殖民地建成后,在完成十万定居者年度供水并开始居住时,水资源损耗、输送成本与所需时间之间的关联关系。
{Data preprocessing}
%数据预处理

Based on research into water consumption standards for urban residents prioritising water conservation, we have selected the average value of the secondary consumption ceiling, namely $215$ litres per day, as the per capita daily water usage. Given that water has a density of 1 tonne per cubic metre, and 1 cubic metre equals $1,000$ litres,

it follows that the annual water requirement for a population of $100,000$ can be calculated as
%基于对城市居民节水优先用水标准的研究,我们选取次级用水上限的平均值——即每日215升——作为人均日用水量。考虑到水的密度为每立方米1吨,且1立方米等于1000升,由此可推算10万人口的年用水需求量为
$
    U_{wather}=7847500
$
The advanced water recycling system on the space station achieves a $90\%$ recovery efficiency. Consequently, the annual volume of water requiring transport for replenishment is
%空间站先进的水循环利用系统技术实现90%的回收效率。因此,每年需运输补充的水量为
$
    U_{loss}=784750
$

\subsubsection{Establishment of a Water Demand Estimation Model}
%水需求估算模型的建立
The application of the three transport models from Question 1 (under idealised perfect conditions) yielded the following conclusions:
%应用问题1中的三种运输模型(在理想化完美条件下)得出以下结论

\begin{itemize}
    \item \textbf{{Solution for scenario i}}
\end{itemize}
The time required for mode of transport A is as follows
%运输方式A所需时间如下
$$%3
    T_a=\frac{U_{loss}}{537000}
$$
The following transport costs are required
%需要支付以下运输费用
$$%4
    D_a=1015U_{loss}
$$
\begin{itemize}
    \item \textbf{{Solution for scenario ii}}
\end{itemize}
The time required for mode of transport B is as follows
%运输方式B所需时间如下
$$%5
    T_b=\frac{U_{loss}}{(100\sim150)\times7300}
$$
The following transport costs are required
%需要支付以下运输费用
$$%6
    D_b=5000U_{loss}
$$
\begin{itemize}
    \item \textbf{{Solution for scenario iii}}
\end{itemize}
The time required for mode of transport C is as follows
%运输方式C所需时间如下
$$%7
    \begin{cases}
	T_c=max\{\frac{M_a}{3 f_ax},\,\frac{M_b}{10 f_bx}\} \\
	N_a+N_b=U_{loss}\\
\end{cases}
$$
The following transport costs are required
%需要支付以下运输费用
$$%8
    D_c=1015W_a+5000W_b
$$
\textbf{The application of the three transport models from Question 2 (in the presence of disturbances) yields the following conclusions}
%应用问题2中的三种运输模型(存在干扰的情况下)得出以下结论
\begin{itemize}
    \item \textbf{{Solution for scenario i}}
\end{itemize}
The time required for mode of transport A is as follows
%运输方式A所需时间如下
$$%3
    T_a=\frac{U_{loss}}{537000\alpha}
$$
The following transport costs are required
%需要支付以下运输费用
$$%4
    D_a=1015U_{loss}
$$
\begin{itemize}
    \item \textbf{{Solution for scenario ii}}
\end{itemize}
The time required for Transport Method B is
%运输方式B所需时间为
$$
    T_b=\frac{U_{loss}}{100\sim150\times7300\beta}
$$
The following transport costs are required
%需要支付以下运输费用
$$
    D_b=5000U_{loss}
$$
\begin{itemize}
    \item \textbf{{Solution for scenario iii}}
\end{itemize}
The time required for mode of transport C is as follows:
%运输方式C所需时间如下
\begin{equation}
    \begin{cases}
	T_c=max\{\frac{M_a}{3\alpha f_ax},\,\frac{M_b}{10\beta f_bx}\} \\                       
	N_a+N_b=U_{loss}\\
   \end{cases}
\end{equation}


The following transport costs are required
%需要支付以下运输费用
$$%8
    D_c=1015W_a+5000W_b
$$
\subsubsection{Solving the model}
Substituting the scenario into all the above equations
$M_{total}=7847500$ ,\,$x=125$ ,\,$\alpha=1$ ,\,$\beta=1$ .
%将该情景代入上述所有方程 $M_{total}=10^8$ ,\,$x=125$ 
\begin{table}[H]
\centering
\renewcommand{\arraystretch}{1.6} 
\caption{Time cost statistics ($\alpha=1$ ,\,$\beta=1$)}
	\begin{tabular}[cp{0.9\textwidth}]{ccc}
		\toprule[2pt]
		proposal &   Delivery time(years) & Delivery cost(\$)  \\
		\toprule[2pt]
        %\midrule
		\textbf{{Solution for scenario i}} & $1.46$ & $7.965\times10^{8}$  \\
        \textbf{{Solution for scenario ii}} & $0.87$ & $3.924\times10^{9}$  \\
        \textbf{{Solution for scenario iii}} & $0.55$ & $2.755\times10^{9}$  \\        
		\bottomrule[2pt]
	\end{tabular}
	\label{tab1} 
\end{table}

Substituting the scenario into all the above equations
$M_{total}=7847500$ ,\,$x=125$ ,\,$\alpha=0.95$ ,\,$\beta=0.98$ .
%将该情景代入上述所有方程 $M_{total}=10^8$ ,\,$x=125$ 
\begin{table}[H]
\centering
\renewcommand{\arraystretch}{1.6} 
\caption{Time cost statistics ($\alpha=0.95$ ,\,$\beta=0.98$)}
	\begin{tabular}[cp{0.9\textwidth}]{ccc}
		\toprule[2pt]
		proposal &   Delivery time(years) & Delivery cost(\$) \\
		%\midrule
        \toprule[2pt]
		\textbf{{Solution for scenario i}} & $1.54$ & $7.965\times10^{8}$ \\
        \textbf{{Solution for scenario ii}} & $0.89$ & $3.924\times10^{9}$  \\
        \textbf{{Solution for scenario iii}} & $0.56$ & $2.778\times10^{9}$  \\        
		\bottomrule[2pt]
	\end{tabular}
	\label{tab1} 
\end{table}
\subsubsection{Analysis of Results}%结论分析
In comparison to the transportation of materials during base construction, the proportion of water transport requirements relative to construction needs is reduced. The pressure on water transport within the hybrid model is less significant. Once a lunar water recycling system is established with a recovery efficiency of $90\%$, and considering only the replenishment of losses, the system can fully accommodate this demand. Relying solely on rocket launch sites incurs higher operational costs, while a space elevator system alone lacks the critical redundancy for uninterrupted life-sustaining supplies. Consequently, the hybrid transport model proves the most stable solution under both ideal conditions and in the presence of disruptive factors. It reliably replenishes water losses within lunar colonies in a timely manner, establishing itself as a dependable transport framework.
%相较于基地建设期间的物资运输,水资源运输需求占建设需求的比例有所降低。混合运输模式下,水资源运输的压力相对较小。一旦建立回收效率达90%的月球水循环系统,且仅考虑损失补给,该系统即可完全满足需求。仅依赖火箭发射场将导致更高运营成本,而单靠太空电梯系统又缺乏保障生命维持物资持续供应的关键冗余机制。因此,混合运输模式在理想条件与干扰因素并存的双重情境下均展现出最稳定的解决方案特性。该模式能可靠及时地补充月球殖民地的水资源消耗,确立其作为可靠运输框架的地位。
%----------------------------------------------------------------------
\subsection{ Analysis of Environmental Impact Assessment and Minimization Strategies}
\subsubsection{Problem Analysis}%问题分析
In order to assess the environmental impact on Earth of establishing a lunar colony with a population of $100,000$, the primary focus is on atmospheric pollution and carbon emissions. These are generated by the transport system during both the construction phase (transporting $100$ million tonnes of materials) and the operational phase (transporting approximately $550,000$ tonnes of water annually). The question explicitly states that the Galactic Port system "does not generate atmospheric pollution". In contrast, traditional rocket launches produce significant emissions. Therefore, this section first quantitatively compares the environmental impacts of three transport options, then discusses how to modify the original model to minimise environmental impact.
%为评估建立10万人口月球殖民地对地球环境的影响,研究重点聚焦于大气污染与碳排放问题。这些污染物主要源于运输系统在建设阶段(运输1亿吨材料)和运营阶段(每年运输约55万吨水)的排放。问题明确指出银河港系统"不会产生大气污染"。而传统火箭发射则会产生大量排放物。因此本节首先定量比较三种运输方案的环境影响,随后探讨如何修改原始模型以最大限度降低环境影响。

A study of the technical characteristics of the two transport systems reveals their primary environmental impact factors to be as follows:

Rocket fuel combustion produces substantial CO₂, water vapour, and nitrogen oxides.

Atmospheric pollution is caused by the emission of particulate matter and chemical pollutants during rocket launches.

 The energy consumption of space elevator systems is dependent on the energy mix, with the system relying on grid electricity.


In order to compare the environmental impacts of different proposals, the following quantitative framework must be established:

    
The use of $CO_2$ equivalents $CO_2e$ as a unified emissions metric is essential.

The construction phase (transporting 100 million tonnes of building materials) is denoted as $M_{i1}$,\,$(i \in \{a, b\})$, while the operational phase (transporting $784750$ tonnes of water annually) is denoted as $M_{i2}$,\,$(i \in \{a, b\}).$

Both emissions intensity per unit transported and total cumulative emissions must be considered.

This analysis is conducted under idealised conditions.
%对两种运输系统技术特征的研究表明,其主要环境影响因素如下
%火箭燃料燃烧会产生大量二氧化碳、水蒸气和氮氧化物。
%大气污染源于火箭发射过程中排放的颗粒物及化学污染物。
%太空电梯系统的能耗取决于能源结构,该系统依赖电网供电。
%为比较不同方案的环境影响,需建立以下量化框架
%采用二氧化碳当量($CO_2e$)作为统一排放计量单位至关重要。
%需区分建造阶段(运输1亿吨建筑材料)记为M_{i1}($i \in \{a, b\}$)与运营阶段(每年运输784750吨水)M_{i2}($i \in \{a, b\}$)。
%必须同时考量单位运输量的排放强度及累计总排放量。
%本分析基于理想化条件进行。
\subsubsection{Establishment of a Quantitative Assessment Model}%量化评估模型建立
The unit carbon emission intensity of Galaxy Harbour is
$0.01 ton CO_2e/KWh$
%银河港的单位碳排放强度为0.01ton CO2/吨

The carbon emissions intensity per rocket unit is
$32 ton CO_2e/ton$
%每枚火箭单位的碳排放强度为32ton CO2/吨

The transportation of 100 million tonnes of building materials is hereby proposed.
    %现提议运输1亿吨建筑材料。


Constraint

$$
    M_{a1}+M_{b1}=M_{total}\,\,\,\,\,M{a1}\geqslant0 \,\,\,\,\,M{b1}\geqslant0
$$

Carbon emissions

$$
    E_c = 0.01\times M_{A1} + 32\times M_{B1}
$$

 Transport $784750$ tonnes of water annually.
    %每年运输78万吨水


$$
    M_{a2}+M_{b2}=784750\,\,\,\,\,M{a1}\geqslant0 \,\,\,\,\,M{b1}\geqslant0
$$

$$
    E_{c water} = 0.01\times M_{A2} + 32\times M_{B2}
$$

Carbon emissions during the construction phase and operational phase

\begin{equation}
    E = E_C + E_{c water} = 0.01\times (M_{A1} + M_{A2}) + 32\times(M_{B1} + M_{B2}) 
\end{equation}
      


\subsubsection{Solving the model}
Substituting the scenario into all the above equations:
$M_{total}=100000000$ ,\,$x=125$ ,\,$\alpha=1$ ,\,$\beta=1$ .
%将该情景代入上述所有方程 $M_{total}=10^8$ ,\,$x=125$ 
\begin{table}[H]
\centering
\renewcommand{\arraystretch}{1.6} 
\caption{Question 4  Cost-Time Carbon Emissions ($\alpha=1$ ,\,$\beta=1$)}
%问题4  成本时间碳排放表
	\begin{tabular}[cp{0.9\textwidth}]{cccc}
		\toprule[2pt]
		proposal &  Total time & Total cost  & Total carbon emissions\\
                & (Years) & (\$)  & (10,000 tonnes $CO_2e$) \\
		%\midrule
        \toprule[2pt]
		\textbf{{Solution for scenario i}} & $187.68$ & $1022.97$ & $100.78$ \\
        \textbf{{Solution for scenario ii}} & $111.98$ & $5039.24$ &  $322511.20$ \\
        \textbf{{Solution for scenario iii}} & $70.14$  & $3538.38$ &  $202027.98$ \\        
		\bottomrule[2pt]
	\end{tabular}
	\label{tab1} 
\end{table}

Substituting the scenario into all the above equations
$M_{total}=100000000$ ,\,$x=125$ ,\,$\alpha=0.95$ ,\,$\beta=00.98$ .
%将该情景代入上述所有方程 $M_{total}=10^8$ ,\,$x=125$ 
\begin{table}[H]
\centering
\renewcommand{\arraystretch}{1.6} 
\caption{Cost-Time carbon emissions ($\alpha=0.95$ ,\,$\beta=00.98$)}
	\begin{tabular}[cp{0.9\textwidth}]{cccc}
		\toprule[2pt]
		proposal &  Total time & Total cost  & Total carbon emissions\\
                & (Years) & (\$)  & (10,000 tonnes $CO_2e$) \\
		\toprule[2pt]
        %\midrule
		\textbf{{Solution for scenario i}} & $197.56$ & $1022.97$ & $100.78$ \\
        \textbf{{Solution for scenario ii}} & $114.27$ & $5039.24$ & $322511.20$ \\
        \textbf{{Solution for scenario iii}} & $72.40$  & $3567.48$ & $204364.68$ \\        
		\bottomrule[2pt]
	\end{tabular}
	\label{tab1} 
\end{table}
\subsubsection{Analysis of Results}%结论分析
A thoroughgoing environmental impact analysis indicates that the three transport options differ in cost and time, and also exert markedly distinct effects upon the Earth's ecosystems. When pollution from the space elevator system is defined solely by the value of electricity generated by the global grid, and rocket transport alone generates substantial pollution per launch, incorporating environmental impact into the objective function reveals a solution. By allocating the majority of material transport to the space elevator while reserving rockets for supplementary and time-sensitive shipments, this approach maintains a reasonable construction schedule while significantly reducing total emissions. It also ensures the system reliability of life-critical supplies such as water. The hybrid transport strategy proves not only economically and temporally optimal, but also environmentally superior.
%全面的环境影响分析表明,三种运输方案在成本和时间上存在差异,对地球生态系统的影响也截然不同。当空间电梯系统的污染仅以全球电网发电价值为衡量标准,而火箭运输每次发射都会产生大量污染时,将环境影响纳入目标函数便能揭示解决方案。通过将大部分物资运输交由太空电梯承担,仅保留火箭用于补充性及时效性运输,该方案既能维持合理的建设进度,又能大幅降低总排放量。同时确保了水等生命关键物资的系统可靠性。这种混合运输策略不仅在经济性与时间效率上达到最优,更在环境效益方面具有显著优势。

\section{Model validation}%模型验证
\subsection{Sensitivity Analysis}
In order to validate the robustness of the model and the reliability of its conclusions, a sensitivity analysis was conducted. This analysis involved systematically varying the payload capacity for each rocket launch, in order to observe its impact on transport costs, time requirements, and the selection of optimal solutions.
%为验证模型的稳健性及其结论的可靠性,进行了敏感性分析。该分析通过系统性地改变每次火箭发射的有效载荷能力,观察其对运输成本、时间要求及最优方案选择的影响。

\begin{figure}[H]
	\small
	\centering
	\includegraphics[width=9.4cm,height=6cm]{picture6.png}
	\caption{The effect of load variation on time}
\end{figure}


A sensitivity analysis of varying payload capacity is hereby presented.
%现对不同有效载荷容量进行敏感性分析。
As demonstrated in the accompanying figure, the discrete enumeration search algorithm, when utilised with a load range of (100–150) and a load increment of 10, establishes a correlation between the duration of Scenario C's transport and the respective tonnage allocated to each transport method. This analysis substantiates that the shortest transport duration is attained when both methods commence and conclude concurrently, with the associated cost falling between the costs incurred when either method is employed independently. When incorporating time costs, the hybrid transport approach still yields the optimal outcome, thereby demonstrating the model's soundness.
%如附图所示,当采用(100–150)的载荷范围和10的载荷增量时,离散枚举搜索算法建立了场景C运输时长与各运输方式对应吨位分配之间的关联性。该分析证实当两种运输方式同时启动并结束时,运输时长达到最短,其成本介于单独采用任一方式所产生的成本之间。即使计入时间成本,混合运输方案仍能实现最优结果,从而验证了模型的可靠性。
\begin{figure}[H]
	\small
	\centering
	\includegraphics[width=9.4cm,height=6cm]{picture1.png}
	\caption{The efectof disturbance coefcientson transport time}
\end{figure}
A sensitivity analysis of disturbance coefficients was conducted, with particular reference to the availability of the Galaxy Port System and the rocket launch success rate. To this end, values for alpha and beta were selected, ranging from the assumed values of $0.95$ and $0.98$, respectively. These values were incremented and decremented in steps of $0.01$ on two occasions. It was observed that this factor influenced specific transport times; however, the overall trend remained unaltered.
%针对扰动系数进行了敏感性分析,重点考察了银河港系统可用性与火箭发射成功率的影响。为此,分别选取了α系数和β系数的取值区间,其初始假设值分别为0.95和0.98。该区间值以0.01为步长进行了两次增减操作。分析发现该因素虽影响特定运输时长,但整体趋势未发生改变。


\subsection{Error Analysis}

Errors arising from the frequency of rocket launches

The frequency of rocket launches assumed in this study is based on the uniformity of global rocket launch capabilities, thus disregarding geographical location. This assumption introduces an error relative to actual conditions. While this error may have a significant impact, under the premise of considering only time costs, the hybrid transport model still achieves the shortest duration. Although numerical results may exhibit deviations due to modelling simplifications, the relative advantage of the hybrid transport strategy remains unaffected.
%因火箭发射频率导致的错误
%本研究假设的火箭发射频率基于全球火箭发射能力的均匀性,因此忽略了地理位置因素。该假设相对于实际情况存在误差。尽管该误差可能产生显著影响,但在仅考虑时间成本的前提下,混合运输模型仍能实现最短时长。虽然数值结果可能因建模简化而出现偏差,但混合运输策略的相对优势依然保持不变。

Cost-related discrepancies

The actual transport costs of space elevator technology remain unknown at present, being based solely on theoretical projections. Furthermore, the calculation of rocket launch costs is derived from the per-kilogram cost of launching rockets into low Earth orbit. Nevertheless, the fact that space elevator technology's transport costs are lower than those of rocket-only transport remains unchanged, and the relative advantages of hybrid transport strategies remain unaffected.
%成本相关差异
%目前空间电梯技术的实际运输成本仍不明确,仅基于理论推测。此外,火箭发射成本的计算源于将火箭送入近地轨道每公斤的成本。然而,空间电梯技术运输成本低于纯火箭运输的事实并未改变,混合运输策略的相对优势也依然存在。
\section{Model evaluation and optimization}%下文将评估该模型的优缺点,并提出改进方案。
\subsection{Advantages of the model}
Possesses versatility:

The established hybrid transport model has been shown to address a range of problems across multiple scenarios. This demonstrates its universality as a parameterised and transferable framework applicable beyond the specific assumptions of this particular problem to diverse transport contexts.

Its robustness under parameter variations and scalability for different types of supply-demand scenarios (construction materials, water, life support resources) attests to its engineering versatility.

The algorithmic fusion achieves high efficiency while balancing precision and performance:

the model does not confine itself to a single algorithm, but instead constructs a hybrid algorithmic framework that successfully integrates the global optimisation capabilities of the particle swarm algorithm, the precision of enumerative methods, and the phased decision-making advantages of dynamic programming.
%具有通用性
%已建立的混合运输模型在多种场景下均展现出解决各类问题的能力。这证明其作为参数化可迁移框架的普适性——该框架不仅适用于特定问题的假设条件,更能推广至多元运输情境。
%其在参数变动下的稳健性,以及对各类供需场景(建筑材料、水资源、生命保障物资)的可扩展性,充分印证了该模型的工程通用性。
%算法融合在兼顾精度与性能的同时实现了高效性该模型不局限于单一算法,而是构建了混合算法框架,成功整合了粒子群算法的全局优化能力、枚举法的精确性,以及动态规划的分阶段决策优势。
\subsection{Limitations of the model}
The cost model is overly simplistic in its approach:

It assumes that the costs of lifts and rockets are fixed values, whereas in reality these costs may vary according to cargo volume, distance, and technological maturity. The model fails to distinguish between fixed costs (such as lift construction and rocket launch site maintenance) and variable costs.

The model does not take into account variations in launch sites:

The model enumerates ten distinct rocket launch sites, which may differ in geographical location, launch frequency, cost, and payload capacity. However, the model treats them as entirely identical.

The model does not reflect coordination between construction and transport schedules.:

The 100 million tonnes of materials would not be transported in a single shipment. Transport may occur in batches, with construction proceeding concurrently. The time model should be more dynamic.

The reliability coefficients $\alpha$ and $\beta$ lack empirical justification. The model extensively employs predictive or hypothetical data.

The model disregards environmental factors such as carbon emissions and energy consumption.

%该成本模型在方法上过于简单化:
%它假设运载工具和火箭的成本是固定值,而实际上这些成本可能因货物体积、距离和技术成熟度而变化。该模型未能区分固定成本(如运载工具建造和火箭发射场维护)与可变成本。
%因火箭发射频率导致的错误:
%该模型未考虑发射基地差异性。
该模型列举了十个不同的火箭发射场,它们在地理位置、发射频率、成本和有效载荷能力方面可能存在差异。然而,该模型将它们视为完全相同的。
%该模型未反映建造与运输计划的协调性:
%1亿吨材料不可能一次性运输完成,运输可能分批次进行,建造工作则可同步推进。时间模型应更具动态性。
%可靠性系数$\alpha \beta$缺乏实证依据。模型大量采用预测性或假设性数据。
%模型忽略碳排放、能耗等环境因素。
\subsection{Model optimisation}
The following paper sets out the introduction of uncertainty modelling and robust optimisation, as well as stochastic programming. In this section, critical uncertain parameters (e.g. rocket transport costs, per capita water consumption) are modelled as random variables, with optimisation targeting the shortest time.

The subsequent section reintroduces robust optimisation by defining fluctuation ranges for uncertain parameters. This is done in order to seek a robust solution that maintains acceptable performance under all worst-case scenarios. For instance, a strategy may be optimised so that transport time remains below a specified threshold even when parameters fluctuate within $\pm5$\%.
%本文阐述了不确定性建模与鲁棒优化以及随机规划的引入。本节中,关键不确定参数(如火箭运输成本、人均用水量)被建模为随机变量,优化目标设定为最短运输时间。
%后续章节通过为不确定参数设定波动范围重新引入鲁棒优化,旨在寻求在所有最坏情境下仍能保持可接受性能的稳健解。例如,可优化运输策略使运输时间始终低于指定阈值,即使参数波动幅度达±5%亦能满足要求。

\section{Correspondence}%信件

\noindent Theme: Proposal for a Sustainable, Phased Logistics Strategy for Lunar Colonization	
\begin{letter}{To MCM:}

\qquad The establishment of a lunar colony comprising $100,000$ worth of inhabitants presents a considerable logistical challenge. The construction phase alone necessitates the transportation of $100$ million worth of materials to the Moon, thereby placing a significant burden on Earth. The subsequent long-term operational phase is even more demanding; according to the model utilised, the annual water consumption exceeds $7,847,500$ tons. Even assuming a $90\%$ recycling rate,$784,750$ tons of makeup water must be transported annually (approximately $2,150$ tons per day). When accounting for other essential living supplies, the total transport volume will be considerably higher.
%建立一个价值10万美元的月球殖民地面临着巨大的后勤挑战。仅建设阶段就需要向月球运输价值1亿美元的物资,这给地球带来了沉重负担。后续的长期运营阶段要求更为严苛根据所采用的模型,年用水量将超过780万吨。即便假设90%的回收率,每年仍需运输78万吨补充水(约合每日2150吨)。若计入其他必需生活物资,总运输量将大幅攀升。

\qquad Core Data Comparison: Environmental Feasibility Analysis
Based on IEA $2050$ grid projections and physical modelling, we compared the carbon intensity of two transport methods:
%核心数据对比环境可行性分析
%基于国际能源署2050年电网预测及物理模型,我们对比了两种交通方式的碳强度


\qquad\textbf{Conventional rockets:}Approximately $32$ tonnes $CO_2$ / tonne payload
Galaxy Port System: Approximately $0.01$ tonnes $CO_2$ / tonne payload
%{传统火箭}约32吨二氧化碳/吨有效载荷

\qquad\textbf{Galaxy Port System:} Approximately $0.01$ tonnes $CO_2$ / tonne payload
%银河港系统约0.01吨二氧化碳/吨有效载荷

\qquad This represents a three-order-of-magnitude difference. The Galaxy Port System, powered electrically, will see its environmental advantages further amplified by advancements in power generation technology. It stands as the sole solution offering long-term economic and environmental sustainability.
%这代表着三个数量级的差异。采用电力驱动的银河港系统,其环境优势将随着发电技术的进步而进一步放大。它作为唯一能提供长期经济与环境可持续性的解决方案而屹立不倒。

\qquad Based on these findings, we propose the following \textbf{Three-Phase Action Plan}. This plan aligns with current technology readiness, provides sufficient construction time for the space elevator, and achieves a gradual, realistic transition:
%基于上述研究结果,我们提出以下\textbf{三阶段行动计划}。该计划既符合当前技术成熟度,又为太空电梯建设提供了充足时间,可实现循序渐进、切实可行的过渡

\qquad \textbf{Phase I: Early Construction Phase.} Given that rocket technology is mature, flexible, and immediately available, rockets will serve as the primary transport mode in this phase. The focus is on transporting foundational infrastructure and equipment to establish the initial lunar base.
%{第一阶段早期建设阶段} 鉴于火箭技术已臻成熟、灵活且可立即投入使用,火箭将成为本阶段的主要运输方式。重点在于运送基础建设物资与设备,以建立初始月球基地。

\qquad \textbf{Phase II: Transitional Phase.} As construction demands grow, a hybrid strategy should be adopted. Rockets—offering speed and flexibility—will continue to handle urgent and special missions. Meanwhile, the Galaxy Port system will begin to assume responsibility for the bulk of material transport, significantly reducing carbon emissions.
%{第二阶段过渡期} 随着建设需求增长,应采取混合策略。火箭凭借其速度与灵活性,将继续承担紧急及特殊任务。与此同时,银河港系统将逐步承担起大部分物资运输任务,从而大幅降低碳排放量。

\qquad \textbf{Phase III: Long-Term Operation Phase.} Once the colony is operational and requires continuous high-volume resupply, the Galaxy Port system should become the main logistical pillar. Its electric drive, extremely low carbon intensity, and suitability for continuous transport make it the only viable long-term solution, conducive to achieving the Net Zero goal.
%{第三阶段长期运营阶段}当殖民地投入运营并需要持续高容量补给时,银河港系统应成为主要后勤支柱。其电力驱动特性、极低的碳强度以及适用于持续运输的优势,使其成为唯一可行的长期解决方案,有助于实现净零排放目标。

\qquad These three phases ensure short-term feasibility, long-term sustainability, and alignment with global decarbonization targets. This represents the optimal action plan for establishing and maintaining a $100,000$-person lunar colony. Therefore, we strongly urge the MCM to adopt this phased transport strategy and prioritize the development of the Galaxy Port system as a core component of future Earth-Moon infrastructure planning.
%这三个阶段确保了短期可行性、长期可持续性,并与全球脱碳目标保持一致。这构成了建立并维持十万人月球殖民地的最优行动方案。因此,我们强烈敦促月球管理委员会采纳这一分阶段运输战略,并将银河港系统开发作为未来地月基础设施规划的核心组成部分予以优先推进。
    
    \vspace{\parskip}
	Sicerely yours,\\
	Your friends
\end{letter}	


\begin{thebibliography}{99}
\bibitem{1}Jones H W. The Recent Large Reduction in Space Launch Cost[J]. Albuquerque, NM: 48th International Conference on Environmental Systems, 2018.

\bibitem{2}Swan P A, Raitt D I, Swan C W, et al. Space Elevators: The Green Road to Space[M]. Paris: International Academy of Astronautics, 2021.

\bibitem{3}Federal Aviation Administration. The Annual Compendium of Commercial Space Transportation: 2024[M]. Washington, DC: FAA Office of Commercial Space Transportation, 2024.

\bibitem{4}Ishimatsu T, de Weck O L, Hoffman J A, et al. Generalized Multi-Commodity Network Flow Model for the Space Supply Chain[J]. Reston, VA: Journal of Spacecraft and Rockets, 2016.

\bibitem{5}SpaceX. Falcon Heavy Users Guide (Revision 3.0)[M]. Hawthorne, CA: Space Exploration Technologies Corp., 2023.
\end{thebibliography}

%\begin{appendices}
%\section{AI Usage Report}%AI使用报告
%\begin{lstlisting}
%clean 
%\end{lstlisting}

%\end{appendices}

\newpage

{\centering  
	\textbf{\underline{Report on use of AI}}  
	\par}  

\begin{enumerate}  
	\item OpenAl \textit{ChatGPT} (Feb 2026 version, GPT-4o)\\
	Query1: \textit{< Is it necessary to acknowledge the use of translation tools? >}\\
	Output: \textit{<Generally, no. The use of pure translation tools (e.g., Google Translate, DeepL) for converting text from one language to another typically does not need to be formally acknowledged in reports or submissions. This is because they are considered aids for language comprehension rather than tools that generate original ideas, content, code, or solutions. >} 
	\item DeepSeek-AI \textit{DeepSeek} (Web Version, Feb 2026, DeepSeek-V3)\\
	Query1: \textit{< Please help me analyze question four. >}\\
	Output: \textit{< The core of Question four lies in assessing the impact of different transportation schemes (only space elevators, only traditional rockets, or a combination of both) on the Earth's environment, and proposing model optimization ideas to minimize this impact. > Adopted: < The impact of different factors on the Earth's environment > }
	\item DeepSeek-AI \textit{DeepSeek} (Web Version, Feb 2026, DeepSeek-V3)\\
	Query1: \textit{< What papers can be found on the availability of space elevators? >}\\
	Output: \textit{< To search for research materials related to the usability of space elevators, you can conduct a search in Chinese literature databases. A direct and effective method is: visit China National Knowledge Infrastructure (CNKI) and search with the following keywords:  Space elevator  Track elevator >}  


\end{enumerate}  
\end{document}
